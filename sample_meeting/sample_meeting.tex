\documentclass{article}
\usepackage[utf8]{inputenc}
\usepackage{amssymb, hyperref}

\title{Sample Meeting}
\author{Diego Alvarez \\ \href{mailto:diego.alvarez@colorado.edu}{diego.alvarez@colorado.edu}}
\date{July 2021}

\begin{document}

\maketitle

\section*{Non-differentialability of Brownian motion}
The fundamental theorem of calculus \emph{breaths life} into calculus. By definition it is
$$
\int_{a}^b f(x) dx = F(b) - F(a)
$$
where $F'(x) = f(x)$
\newline
\newline
This equation allows for the operators used in calculus to be applied such as the derivative and the integral.
\newline
\newline
We'll go over how this theorem doesn't hold for Brownian motions. We can write the definition of the integral as
$$
\int_a^b f(x)dx = \lim_{n \to \infty} \sum_{i=1}^\infty f(x_i^*) \Delta x
$$
Really what we are \emph{hoping} for is that it converges to some value
$$
\lim_{n \to \infty} \sum_{i=1}^\infty f(x_i^*) \Delta x \in \mathbb{R}
$$
The problem is that for a Brownian motion it doesn't converge to a value. That is because when you zoom into a smooth function it gets flat. That's how you can approximate it with rectangles. But you can infinitely zoom into a Brownian motion and it will look bumpy.
\end{document}
